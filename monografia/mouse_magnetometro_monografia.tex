\documentclass[brazil,pagestart=firstchapter]{abnt}
\usepackage[utf8]{inputenc}
\usepackage[brazil]{babel}

% http://tex.stackexchange.com/questions/664/why-should-i-use-usepackaget1fontenc
% http://texblog.net/latex-archive/fonts/symbols/
\usepackage[T1]{fontenc}

% 'microtype' improves LaTeX line-breaking algorithm by using
% microtypographic features of the font
% http://tex.stackexchange.com/questions/349/what-is-the-practical-difference-between-latex-and-pdflatex/358#358
\usepackage{microtype}

% 'hyperref' for including hyperlinks to the PDF.
% http://en.wikibooks.org/wiki/LaTeX/Formatting#Typesetting_URLs
\usepackage{hyperref}
\hypersetup{
	pdfborder={0 0 0}
}


%%%%%%%%%%%%%%%%%%%%%%%%%%%%%%%%%%%%%%%%%%%%%%%%%%%%%%%%%%%%
% Other packages

% Allow including graphics
\usepackage{graphicx}

% To find the total number of pages
%\usepackage{lastpage}

% List of abbreviations
% http://en.wikibooks.org/wiki/LaTeX/Indexing#Abbreviation_list
% http://franz.kollmann.in/latex/latex.html#abbr
%\usepackage{nomencl}
%\makenomenclature
% Insert the abbreviation in the tex as short and long form: 
% \nomenclature{Abbr}{Abbreviation}
%
% latexmk needs a custom dependency for this package
% http://magic.aladdin.cs.cmu.edu/2007/11/06/continuous-latex-compilation-using-latexmk/

%%%%%%%%%%%%%%%%%%%%%%%%%%%%%%%%%%%%%%%%%%%%%%%%%%%%%%%%%%%%
% Embedding source-code

% For inserting source-code
%\usepackage{listings}
% TODO: configure this


%%%%%%%%%%%%%%%%%%%%%%%%%%%%%%%%%%%%%%%%%%%%%%%%%%%%%%%%%%%%
% Table-related packages:

% For multirow cells inside tabular environments
%\usepackage{multirow}

% Longtable allows you to write tables that continue to the next page
%\usepackage{longtable}

% 'tabularx' package - simple column stretching
% http://en.wikibooks.org/wiki/LaTeX/Tables#The_tabularx_package_-_simple_column_stretching
%\usepackage{tabularx}

% 'booktabs' - Publication quality tables in LaTeX
%\usepackage{booktabs}

% Alternating row colors in tables
%\usepackage[table]{xcolor}
%\definecolor{tabular-odd-color}{gray}{0.90}
%\definecolor{tabular-even-color}{gray}{0.97}


%%%%%%%%%%%%%%%%%%%%%%%%%%%%%%%%%%%%%%%%%%%%%%%%%%%%%%%%%%%%
% Fancy headers (and footers)
% http://en.wikibooks.org/wiki/LaTeX/Page_Layout#Page_Styles
%\usepackage{fancyhdr}
%\pagestyle{fancy}
%
%% Headers and footers definition:
%\lhead{\includegraphics[height=21mm]{shared/2aliancas_h.pdf}}
%\chead{}
%% This \parbox is a hack to keep text aligned at top, but it's not the
%% only available solution:
%% http://tex.stackexchange.com/questions/2440/how-to-vertically-align-headers-footers-in-fancyhdr-package
%\rhead{\parbox[b][21mm][t]{0.65\textwidth}{\raggedleft\large\titletext \\[1em] \subtitletext}}
%\lfoot{}
%\cfoot{}
%\rfoot{Page \thepage{} of \pageref{LastPage}}
%
%\renewcommand{\headrulewidth}{0pt}  % Default is 0.4pt
%\renewcommand{\footrulewidth}{0pt}  % Default is 0pt


%%%%%%%%%%%%%%%%%%%%%%%%%%%%%%%%%%%%%%%%%%%%%%%%%%%%%%%%%%%%
% Some metadata

% The following are used at the first page, and also by hypersetup
\newcommand{\titletext}{Mouse USB usando magnetômetro}
\newcommand{\authortext}{Denilson Figueiredo de Sá}

% This is used only by latex-beamer package:
%\title{Mouse USB usando magnetômetro}
%\author{Denilson Figueiredo de Sá}
%\date{2011-11-??}
%\institute{DCC/UFRJ}
%\keywords{django, south, database migration}

% This is needed when not using latex-beamer:
\hypersetup{
	pdftitle={\titletext},
	pdfauthor={\authortext}
}

\begin{document}

% The "abnt" class issues a warning:
%   pdfTeX warning (ext4): destination with the same identifier
%   (name{page.i}) has been already used, duplicate ignored
% This page has a solution (or workaround):
% http://en.wikibooks.org/wiki/LaTeX/Hyperlinks#Problems_with_Links_and_Pages
%\pagenumbering{roman}
% But, anyway, it's better to ignore that warning (or even better would be
% to report it to abntex maintainers).
% Instead of messing with page numbering, I've added pagestart=firstchapter
% option.

\autor{\authortext}
\titulo{\titletext}
% huh... como inserir o subtítulo?
\orientador{Nelson Quilula Vasconcelos}
\comentario{Monografia apresentada para obtenção do Grau de Bacharel em Ciência da Computação pela Universidade Federal do Rio de Janeiro.}
\instituicao{Departamento de Ciência da Computação \par Instituto de Matemática \par Universidade Federal do Rio de Janeiro}
\local{Rio de Janeiro - RJ, Brasil}
\data{??/11/2011}

\capa

\folhaderosto


% TODO: ver sobre a folha de aprovação, se ela será gerada usando LaTeX ou
% se será inserida usando outra forma.


\begin{resumo}
O resumo será escrito aqui.
\end{resumo}

\begin{abstract}
This will be the abstract, someday.
\end{abstract}


%\chapter*{Dedicatória}
% TODO: escrever dedicatória (opcional)

%\chapter*{Agradecimentos}
% TODO: escrever agradecimentos


\listoffigures

%\listoftables

%\pretextualchapter{Lista de abreviaturas e siglas}

\tableofcontents{}


\chapter{Introdução\label{cap:introducao}}

blhbalahblahsd
a
da
hds
ahd

\end{document}

% vim:filetype=tex ts=4 sw=4 noet tw=76 foldmethod=marker foldmarker=\\begin,\\end foldcolumn=4 foldlevel=1

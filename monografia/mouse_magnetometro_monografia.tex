\documentclass[brazil,pagestart=firstchapter]{abnt}

% This does not accept all Unicode chars (for instance: I²C gives an error)
%\usepackage[utf8]{inputenc}

% This adds support for Unicode chars.
\usepackage{ucs}
\usepackage[utf8x]{inputenc}

\usepackage[brazil]{babel}

% http://tex.stackexchange.com/questions/664/why-should-i-use-usepackaget1fontenc
% http://texblog.net/latex-archive/fonts/symbols/
\usepackage[T1]{fontenc}

% Use another font:
% http://tex.stackexchange.com/questions/553/what-packages-do-people-load-by-default-in-latex/951#951
%\usepackage{lmodern}

% 'microtype' improves LaTeX line-breaking algorithm by using
% microtypographic features of the font
% http://tex.stackexchange.com/questions/349/what-is-the-practical-difference-between-latex-and-pdflatex/358#358
\usepackage{microtype}


%%%%%%%%%%%%%%%%%%%%%%%%%%%%%%%%%%%%%%%%%%%%%%%%%%%%%%%%%%%%
% Other packages

% From abntex:
\usepackage{tabela-simbolos}

% Allow including graphics
\usepackage{graphicx}

% SI Units
\usepackage{siunitx}
\sisetup{
	list-final-separator = { e },
	range-phrase = { a },
	output-decimal-marker = {,}
}

% To find the total number of pages
%\usepackage{lastpage}

% List of abbreviations
% http://en.wikibooks.org/wiki/LaTeX/Indexing#Abbreviation_list
% http://franz.kollmann.in/latex/latex.html#abbr
%\usepackage{nomencl}
%\makenomenclature
% Insert the abbreviation in the tex as short and long form: 
% \nomenclature{Abbr}{Abbreviation}
%
% latexmk needs a custom dependency for this package
% http://magic.aladdin.cs.cmu.edu/2007/11/06/continuous-latex-compilation-using-latexmk/

%%%%%%%%%%%%%%%%%%%%%%%%%%%%%%%%%%%%%%%%%%%%%%%%%%%%%%%%%%%%
% Embedding source-code

% For inserting source-code
%\usepackage{listings}
% TODO: configure this

% Settings I had from another project:
%% Setting the default listings font size
%\lstset{
%	basicstyle=\ttfamily\footnotesize
%}
%
%% AVISO!!!
%% Dentro dos exemplos de código-fonte abaixo, coloquei um "tab" de
%% indentação por estar dentro de um "frame", e "espaços" para a
%% indentação do código de exemplo. Isto foi necessário porque os tabs
%% estavam sendo ignorados dentro do códigos de exemplo.
%
%\lstnewenvironment{shellcode}[1][]
%{\lstset{language=bash,
%	basicstyle=\ttfamily\footnotesize,
%	escapeinside={(*@}{@*)},
%	breaklines=true,
%	breakatwhitespace=true,
%	showspaces=false,
%	showstringspaces=false,
%	frame=shadowbox,
%	rulecolor=\color{black},
%	rulesepcolor=\color{black},
%	#1}
%}{}
%
%\lstnewenvironment{pythoncode}[1][]
%{\lstset{language=Python,
%	basicstyle=\ttfamily\footnotesize,
%	escapeinside={(*@}{@*)},
%	%numbers=left,
%	breaklines=true,
%	breakatwhitespace=true,
%	showspaces=false,
%	showstringspaces=false,
%	frame=shadowbox,
%	frameround=rrrt,
%	rulecolor=\color{black},
%	rulesepcolor=\color{gray},
%	#1}
%}{}


%%%%%%%%%%%%%%%%%%%%%%%%%%%%%%%%%%%%%%%%%%%%%%%%%%%%%%%%%%%%
% Table-related packages:

% For multirow cells inside tabular environments
%\usepackage{multirow}

% Longtable allows you to write tables that continue to the next page
%\usepackage{longtable}

% 'tabularx' package - simple column stretching
% http://en.wikibooks.org/wiki/LaTeX/Tables#The_tabularx_package_-_simple_column_stretching
%\usepackage{tabularx}

% 'booktabs' - Publication quality tables in LaTeX
%\usepackage{booktabs}

% Alternating row colors in tables
%\usepackage[table]{xcolor}
%\definecolor{tabular-odd-color}{gray}{0.90}
%\definecolor{tabular-even-color}{gray}{0.97}


%%%%%%%%%%%%%%%%%%%%%%%%%%%%%%%%%%%%%%%%%%%%%%%%%%%%%%%%%%%%
% Fancy headers (and footers)
% http://en.wikibooks.org/wiki/LaTeX/Page_Layout#Page_Styles
%\usepackage{fancyhdr}
%\pagestyle{fancy}
%
%% Headers and footers definition:
%\lhead{\includegraphics[height=21mm]{shared/2aliancas_h.pdf}}
%\chead{}
%% This \parbox is a hack to keep text aligned at top, but it's not the
%% only available solution:
%% http://tex.stackexchange.com/questions/2440/how-to-vertically-align-headers-footers-in-fancyhdr-package
%\rhead{\parbox[b][21mm][t]{0.65\textwidth}{\raggedleft\large\titletext \\[1em] \subtitletext}}
%\lfoot{}
%\cfoot{}
%\rfoot{Page \thepage{} of \pageref{LastPage}}
%
%\renewcommand{\headrulewidth}{0pt}  % Default is 0.4pt
%\renewcommand{\footrulewidth}{0pt}  % Default is 0pt


%%%%%%%%%%%%%%%%%%%%%%%%%%%%%%%%%%%%%%%%%%%%%%%%%%%%%%%%%%%%
% 'hyperref' for including hyperlinks to the PDF.
% It should be loaded last.
% http://en.wikibooks.org/wiki/LaTeX/Formatting#Typesetting_URLs
\usepackage{hyperref}
\hypersetup{
	pdfborder={0 0 0},
	hyperindex=false
}
% hyperindex=false is required by "tabela-de-simbolos"

% http://en.wikibooks.org/wiki/LaTeX/Labels_and_Cross-referencing#Issues_with_links_to_tables_and_figures_handled_by_hyperref
%\usepackage[all]{hypcap}


%%%%%%%%%%%%%%%%%%%%%%%%%%%%%%%%%%%%%%%%%%%%%%%%%%%%%%%%%%%%
% Custom commands

%\newcommand{\something}{something else}

%%%%%%%%%%%%%%%%%%%%%%%%%%%%%%%%%%%%%%%%%%%%%%%%%%%%%%%%%%%%
% Some metadata

% This is used only by latex-beamer package:
%\title{Mouse USB usando magnetômetro}
%\author{Denilson Figueiredo de Sá}
%\date{2011-11-16}
%\institute{DCC/UFRJ}
%\keywords{AVR, USB, mouse, magnetometer}

\autor{Denilson Figueiredo de Sá}
\titulo{Mouse USB usando magnetômetro}
\orientador{Nelson Quilula Vasconcelos}
%\comentario{}
\instituicao{Departamento de Ciência da Computação \par Instituto de Matemática \par Universidade Federal do Rio de Janeiro}
\local{Rio de Janeiro - RJ, Brasil}
\data{16 de novembro de 2011}

% latex-beamer sets the pdftitle and pdfauthor automatically, but here we
% must explicitly run it:
\hypersetup{
	pdftitle={\ABNTtitulodata},
	pdfauthor={\ABNTautordata}
}

%%%%%%%%%%%%%%%%%%%%%%%%%%%%%%%%%%%%%%%%%%%%%%%%%%%%%%%%%%%%
\begin{document}

% The "abnt" class issues a warning:
%   pdfTeX warning (ext4): destination with the same identifier
%   (name{page.i}) has been already used, duplicate ignored
% This page has a solution (or workaround):
% http://en.wikibooks.org/wiki/LaTeX/Hyperlinks#Problems_with_Links_and_Pages
%\pagenumbering{roman}
% But, anyway, it's better to ignore that warning (or even better would be
% to report it to abntex maintainers).
% Instead of messing with page numbering, I've added pagestart=firstchapter
% option.


\capa

\folhaderosto


\begin{folhadeaprovacao}

\setlength{\ABNTsignthickness}{0.4pt}
\setlength{\ABNTsignskip}{2cm}
\hspace*{1cm}

\centerline{\textbf{\large \ABNTtitulodata}}

\bigskip
\bigskip

\centerline{\textbf{\ABNTautordata}}

\bigskip
\bigskip

Projeto Final de Curso submetido ao Departamento de Ciência da Computação
do Instituto de Matemática da Universidade Federal do Rio de Janeiro como
parte dos requisitos necessários para obtenção do grau de Bacharel em
Ciência da Computação.

Apresentado por:

\assinatura{\ABNTautordata}

Aprovado por:

\assinatura{Prof. Nelson Quilula Vasconcelos \\ Orientador}
\assinatura{Prof. Adriano Joaquim}
\assinatura{Profª. Silvana Rossetto}

\bigskip
\bigskip
\bigskip

\begin{center}
\ABNTlocaldata

\ABNTdatadata
\end{center}

\end{folhadeaprovacao}


%\chapter*{Dedicatória}
% TODO: escrever dedicatória (opcional)

%\chapter*{Agradecimentos}
% TODO: escrever agradecimentos


\begin{resumo}
O resumo será escrito aqui.
\end{resumo}

\begin{abstract}
This will be the abstract, someday.
\end{abstract}


\listadesiglas

\listoffigures

%\listoftables

%\pretextualchapter{Lista de abreviaturas e siglas}

\tableofcontents{}


\chapter{Introdução\label{cap:introducao}}

Falar sobre: motivação, resumo, possíveis aplicações (palestras,
necessidades especiais, interatividade em jogos e exposições)

\section{Trabalhos relacionados\label{sec:trabalhos_relacionados}}

TrackIR, Wiimote, eye-tracking


\chapter{Protocolos e componentes\label{cap:protocolos_e_componentes}}

\section{USB\label{sec:usb}}

Descrição resumida de como USB funciona, quais as características e
limitações.
\sigla{USB}{Universal Serial Bus}

\section{USB HID\label{sec:usb_hid}}

Descrição do USB HID.
%\sigla{HID}{Human Interface Device}

\section{Protocolo I²C\label{sec:i2c}}

Breve descrição do protocolo I²C.
%\sigla{I2C}{Inter-Integrated Circuit}
%\sigla{TWI}{Two-Wire Interface}

\section{Microcontrolador ATmega8\label{sec:atmega8}}

ATmega8 é um microcontrolador de 8 bits da família AVR, fabricado pela Atmel
Corporation. Suas características principais são \cite{ATmega8}:

\begin{itemize}
\item Arquitetura Harvard, com espaços de endereçamento distintos para 
instruções e para variáveis.
\item \num{8192} bytes de memória \textit{Flash} para guardar o programa.
\item \num{512} bytes de memória EEPROM para guardar parâmetros de configuração do
programa.
\item \num{1024} bytes de memória SRAM para as variáveis.
\item Voltagem de operação de \SIrange{4.5}{5.5}{\volt}.
\item Clock máximo de \SI{16}{\mega\hertz}.
\item Interface de comunicação serial I²C (com o nome de TWI).
\end{itemize}

Descrição resumida das características do microcontrolador ATmega8.
Basicamente, o resumo do datasheet.

\section{Sensor HMC5883\label{sec:sensor}}

Descrição de como o sensor funciona.


\chapter{Descrição do hardware\label{cap:hardware}}

Explicar de maneira geral as partes do circuito.

Falar sobre microcontrolador, cristal, LEDs e botões.

Inserir aqui o diagrama do circuito.

\section{Interface USB\label{sec:hardware_usb}}

Comunicação entre a USB e o microcontrolador.

\section{Interface com o sensor\label{sec:hardware_sensor}}

Problemas com I²C. ``Fonte'' 3V3 para o sensor. Level shifting para I²C.


\chapter{Descrição do software\label{cap:software}}

\section{Bootloader\label{sec:bootloader}}

\section{Comunicação I²C/TWI\label{sec:twi}}

\section{Driver V-USB\label{sec:vusb}}

Falar aqui também sobre a implementação de um teclado. Talvez numa
subsection.

\section{Menu de configuração\label{sec:menu}}

Talvez falar sobre o teclado aqui dentro, não sei.

\section{Transformação de coordenadas\label{sec:coordenadas}}

Falar sobre: ferramentas auxiliares, teoria, resultados.

Falar sobre calibração dos cantos.

Falar sobre calibração do zero (ou deixar para falar isso depois).

\section{Mouse USB\label{sec:mouse}}

Falar sobre a implementação do mouse (que na verdade é absolute pointing
device).

Talvez aqui, talvez numa outra (sub)seção, falar sobre o fluxo da main().

\section{Outros problemas\label{sec:outros_problemas}}

\subsection{Debouncing dos botões\label{sec:debouncing}}

\subsection{Tamanho do firmware\label{sec:firmware_size}}

\subsection{Gravar configurações na EEPROM\label{sec:eeprom}}

Falar sobre gravar na EEPROM de maneira não bloqueante.

\chapter{Conclusões\label{cap:conclusoes}}

Resultados alcançados. O que deu certo. O que não deu certo.

\section{Trabalhos futuros\label{sec:trabalhos_futuros}}

Listar o que pode ser feito a partir deste projeto.


%%fakesection  Bibliografia

\bibliographystyle{abnt-num}
\bibliography{mouse_magnetometro_monografia}

% How to add \url{} to "url=" entries from Pybliographic:
% :%s/\(url\s*=\s*{\)\([^\\].*[^}]\)\(},\?\)/\1\\url{\2}\3/

%%fakesection  Anexo
\anexo

\end{document}

% vim:filetype=tex ts=4 sw=4 noet tw=76
